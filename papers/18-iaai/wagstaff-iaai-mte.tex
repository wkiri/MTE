\def\year{2018}\relax
%File: formatting-instruction.tex
\documentclass[letterpaper]{article} %DO NOT CHANGE THIS
\usepackage{aaai18}  %Required
\usepackage{times}  %Required
\usepackage{helvet}  %Required
\usepackage{courier}  %Required
\usepackage{url}  %Required
\usepackage{graphicx}  %Required
\frenchspacing  %Required
\setlength{\pdfpagewidth}{8.5in}  %Required
\setlength{\pdfpageheight}{11in}  %Required
\copyrighttext{Copyright 2017, California Institute of Technology. 
Government sponsorship acknowledged.}
%PDF Info Is Required:
  \pdfinfo{
/Title (Extracting Information from Scientific Publications for
Planetary Science)
/Author (Kiri L. Wagstaff, Raymond Francis, Thamme Gowda, Nina L. Lanza,
You Lu, Ellen Riloff, Karanjeet Singh)}
\setcounter{secnumdepth}{0}  
 \begin{document}
% The file aaai.sty is the style file for AAAI Press 
% proceedings, working notes, and technical reports.
%
\title{Extracting Information from Scientific Publications for
Planetary Science}
\author{
Kiri L. Wagstaff$^1$,
Raymond Francis$^1$,
Thamme Gowda$^{1,2}$,
Nina L. Lanza$^3$,\\
{\Large \bf You Lu$^1$,
Ellen Riloff$^4$, and
Karanjeet Singh$^{1,5}$}\\
$^1$Jet Propulsion Laboratory, California Institute of Technology,
4800 Oak Grove Drive, Pasadena, CA 91109\\
\{firstname.lastname\}@jpl.nasa.gov\\
$^2$Information Sciences Institute, University of Southern
California,
4676 Admiralty Way \#1001, Marina Del Rey, CA 90292\\
tg@isi.edu
}
\maketitle
\begin{abstract}
We have constructed an information extraction system called the Mars
Target Encyclopedia (MTE) that takes in planetary science publications
and extracts scientific knowledge.  The extracted knowledge is stored
in a searchable database that can greatly accelerate the ability of
scientists to compare new discoveries with what is already known.  To
date, we have applied this system to $\sim$6000 documents and achieved
XX\% precision in the extracted information. 
\end{abstract}

\section{Introduction}

Scientists everywhere are overwhelmed by the stream of new information
that is published by domain-relevant conferences, workshops, and
journals.  It is difficult both to come up to speed in a new area and
to stay current with the latest discoveries.  In planetary
exploration, new discoveries are potentially made each time new data
is transmitted.  For example, our rovers on Mars have sent back
compositional data for thousands of individual targets (e.g., rocks,
soils), and some of those observations have transformed our
understanding of past environments on the planet {\bf
[cite... Nature?]}. 

To interpret new observations correctly, it is necessary to be able to
compare them with what is already known.  For example, if we observe
high manganese content as a particular location, we want to know if it
is consistent with previous observations or indicates an anomalous new
discovery.  However, no central database exists in which you can query...

We have created a system that uses information extraction methods to
analyze planetary science publications and identify compositional
relationships between Mars surface targets and elements or minerals.
The extracted information is stored in a searchable database that
allows users to ask questions such as ``Which targets contain
hematite?'' or ``What is known about target Dillinger?''  It also
enables entirely new kinds of information visualization, such as a map
display of all locations where the Mars rover Curiosity has detected
hematite. 

Because Mars surface targets follow no standard naming convention,
they can be very difficult to identify in natural text.  Names are
borrowed from Earth locations or people, and a sampling of the names
hints at the challenge of accurately detecting them: ``Dunkirk'',
``Ithaca'', ``Jake'',  ``Old\_woman'', ``Pistol''.  Keyword-based
searches return many spurious hits, and no tool yet exists...


... provide guidance for next steps in exploration 
- hypothesis generation and data collection.

leveraged a small amount of hand-labeled text to train a

precision more important than recall

\section{Related Work}

GeoDeepDive

work at ISI?

\section{Machine Learning for Information Extraction}

{\bf [system diagram from PPT?]}

{\bf Text extraction.}  We used the Apache Tika
parser~\cite{mattmann:tika11} to convert the source PDF files into
UTF-8 text.

Reference extraction.

We focus on the extraction of information about targets identified by
the ChemCam instrument on the Mars Science Laboratory rover.  ChemCam
obtains compositional spectra from up to seven meters away from a
given target, and the resulting spectrum can be analyzed to identify
individual elements within the target~\cite{maurice:chemcam12}.  As of
sol 1159, ChemCam had observed more than 1100 distinct targets.

\subsection{Named Entity Recognition}

- CoreNLP NER CRF
- gazettes - number of items in each (target, element, mineral)

{\bf [gazettes are available where online]}

Target:


- Basilisk gazettes

\subsection{Relation Extraction}

- jSRE SVM

\section{Experimental Results}

\subsection{Corpus}

Our corpus consists of two-page extended abstracts (in PDF format) that
were published by the Lunar and Planetary Science conference (LPSC).  We
identified and manually annotated 113 {\bf [check]} documents from
LPSC 2015 and 2016 that mentioned ``ChemCam.''  We used the 2015
documents for training and the 2016 documents for testing.

{\bf [annotated docs are available where?]}

\subsection{NER Results}

\begin{table}
\caption{Classification accuracy on Mars rover surface images.}
\label{tab:msl}
\begin{center}
\begin{tabular}{l|ll}
Classifier & Train ($n=$) & Test ($n=$) \\ \hline
Random & & \\
Most common & & \\
MSLNet & & \\ \hline
\end{tabular}
\end{center}
\end{table}

{\bf [precision/recall per class - bar plot?]}

\subsection{Relation Extraction Results}

\begin{table}
\caption{Classification accuracy on Mars orbital images.}
\label{tab:hirise}
\begin{center}
\begin{tabular}{l|ll}
Classifier & Train ($n=$) & Test ($n=$) \\ \hline
Random & & \\
Most common & & \\
HiRISENet & & \\ \hline
\end{tabular}
\end{center}
\end{table}

[precision/recall per class - bar plot?]

\subsection{Large-scale Evaluation}

We collected all 6000 {\bf [check]} LPSC documents that were published
in 2014, 2015, and 2016 and ingested them into the MTE.

{\bf [table of num docs, num NER, num RE, time consumed]}

{\bf [Nina's evaluation of extracted information}

\subsection{Limitations}
- no sentence-crossing relationships (about 30\%? recalc this number)
- NER cannot handle overlapping annotations (e.g., calcium sulfate)

\section{Conclusions and Next Steps}
% and lessons learned

This work lies at the intersection of information extraction, machine
learning, and planetary science.  

The MTE is not comprehensive.  There may be compositional information
that was never written up in a scientific publication and therefore
would not be included in the MTE.  Instead, the MTE extracts and
indexes only the information that was judged by scientists to be
worthy of publication to the scientific community.  The MTE leverages
and mirrors this selection bias, and its holdings (like the source
publications) contain only the most valuable and salient information.

\section{Acknowledgments}
This research was carried out in part at the Jet Propulsion Laboratory,
California Institute of Technology, under a contract with the National
Aeronautics and Space Administration.  {\bf [acks for Ellen, Nina?]}

\bibliography{mte}
\bibliographystyle{aaai}
\end{document}
